% lintrans - The linear transformation visualizer
% Copyright (C) 2021-2022 D. Dyson (DoctorDalek1963)

% This program is licensed under GNU GPLv3, available here:
% <https://www.gnu.org/licenses/gpl-3.0.html>

\documentclass[../development.tex]{subfiles}

\begin{document}

\subsubsection{Fixing slots and signals\label{development:preparing-for-v0.2.1:fixing-slots-and-signals}}

I was perusing the Qt5 documentation when I learned about the difference between the \pyinline{dialog.finished} signal and the \pyinline{dialog.accepted} signal. I decided to rework some old code to make better use these signals.

When defining a new matrix or dialog settings, we only want to save the new data if the user actually accepted the dialog by clicking the confirm button. We don't want to save it if they clicked cancel. However, in the case of the error message dialog, we always want to update the render buttons when it's closed, no matter how the user closed the dialog.

%: 66242465222a153a5f37c4a1a3c2bd50bfd90933
%: src/lintrans/gui/main_window.py:35,447-448,461-469,478-483,493,511-512 noscopes

I also added the \pyinline{@pyqtSlot()} decorator to all the relevant methods in the matrix definition dialogs. The types in the brackets indicate the signature of the method.

A slot in Qt5 is just a method that is expected to be connected to a signal, so it gets called from the event loop. Using the decorator makes it clear that a method is a slot, and also allows slightly better performance.

%: 9beff9cf25d3af655e134205572a5668279f42cc
%: src/lintrans/gui/dialogs/define_new_matrix.py:67,138-140,143-144,155-157,164,194-195,203-204,214-215,226,276-277,288-289,305-306,317,348-349,356-357,364-365 noscopes

\end{document}
